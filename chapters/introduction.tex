\section{Context and Motivation}

\paragraph{}Simultaneous Localization and Mapping (SLAM) is one of the most studied topics in robotics~\cite{becnkis2024survey, lajoie2022towards}. Its purpose is to simultaneously estimate the robot's position and orientation (pose) and map the robot's environment. \ac{SLAM} methods often integrate data from devices like \ac{LiDAR}, cameras, or ultrasonic sensors with algorithms like Kalman filters and particle filters, or more advanced and recent approaches, such as Visual~\cite{kazerouni2022survey} and \ac{LiDAR} \ac{SLAM}~\cite{li2025review}. These methods are crucial for applications in Robotics, such as autonomous navigation.

\paragraph{}The accuracy of a \ac{SLAM} method is usually evaluated using the \ac{APE} and the \ac{RPE}. The \ac{APE} refers to the absolute difference between the estimated pose and the ground truth, while the \ac{RPE} refers to the difference between consecutive estimated poses and the respective ground truth of consecutive poses. \ac{SLAM}'s efficacy is dependent on various factors, such as sensor quality, available computational resources, loop closure and algorithm type. The algorithm type is particularly important because different scenarios demand different types of \ac{SLAM} methods. On the other hand, different \ac{SLAM} methods have different number of hyperparameters, impacting the size of the parameter space and the difficulty of arriving at the optimal hyperparameter configuration for a given scenario.

\paragraph{}One key challenge in \ac{SLAM} research is determining the best hyperparameter configuration for a given dataset. Our research group has been developing named \ac{RUSTLE} that streamlines a process, allowing different \ac{SLAM} methods to be run asynchronously, while giving performance reports, allowing for the manual tuning of \ac{SLAM}'s hyperparameters.

% A tool like \ac{RUSTLE} streamlines the process

\paragraph{}There is, however, a missing and crucial component from frameworks like \ac{RUSTLE}: in the literature, \ac{SLAM} methods are usually not optimally tuned, or the comparisons are performed between methods that are close to optimal tuning and methods that are tuned merely to achieve sufficiently satisfactory results, and therefore any comparisons cannot be considered fair and conclusive. Applying hyperparameter optimization algorithms and automating the tuning process would not only relieve the user of the laborious process of manually searching the parameter space for an optimal configuration, but also allow for a more fair comparison between \ac{SLAM} methods.

% or the comparisons are between methods that are almost optimally tuned and methods that are tuned just enough to give sufficiently satisfactory results

\section{Main Goals}

\paragraph{}The main goals of this dissertation are:

\begin{itemize}

    \item Development of an automatic Hyperparameter Tuning (HPT) framework within \ac{RUSTLE} to better optimize and compare the performance of multiple \ac{SLAM} solutions.
    
    \item Assessment of the impact of different hyperparameters have on several SLAM methods.
    
    %\item Get proficient at Rust programming, which not only will help with the development of the previously mentioned framework, but could also be of use in future works.
    
    \item Evaluation of the efficiency of the developed optimization algorithms and compare them to each other.
    
    \item Consolidated overview of the developed work and identify lessons learned and potential future improvements.
    
\end{itemize}

\section{Document Overview}

\paragraph{}The remainder of this dissertation is organized as follows. In chapter 2, essential background concepts and related works are outlined. In chapter 3, a description of the design and implementation of the tuning module developed for \ac{RUSTLE} is provided. In chapter 4, presents the tests results, as well as a brief discussion of those results. In chapter 5, the work is briefly summarized, highlighting its successes and weaknesses, while suggesting avenues for future improvements.