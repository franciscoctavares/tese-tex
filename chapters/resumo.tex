\paragraph{}A parametrização de soluções de Localização e Mapeamento Simultâneo(SLAM) tem potenciais benefícios para aplicações de robótica móvel em diversos ambientes. Embora cada técnica de optimização tenha as suas limitações introduza novos desafios(especificamente em relação aos hiperparâmetros), podem ser ferramentas valiosas no estudo da influência dos parâmetros de um sistema SLAM, bem como na optimização da trajetória e redução de erro em aplicações de robótica móvel. 

\paragraph{}Este trabalho consiste no estudo do estado da arte relativo à optimização de hiperparâmetros e na implementação e teste de três algoritmos de optimização de hiperparâmetros: Pesquisa em grelha(Grid Search), Pesquisa aleatória(Random Search) e Recuo Simulado(Simulated Anealing). Os algoritmos foram escolhidos tendo em conta a diversidade de algoritmos estudados e o tempo disponível para a sua implementação. Os algoritmos podem ser usados para optimizar qualquer solução de \ac{SLAM} disponível, embora os testes sejam feitos com faster-lio e fast-livo2, 2 soluções de \ac{SLAM} utilizadas pela comunidade.

\paragraph{}Neste documento é apresentada uma revisão do estado da arte, identificando
as principais lacunas de pesquisa e fornecendo uma contribuição científica com o
desenvolvimento deste trabalho. Os algoritmos desenvolvidos foram testados em duas soluções de \ac{SLAM}, e os resultados são apresentados e discutidos. Por último, são apresentadas sugestões para melhorias futuras.%A técnica proposta foi testada %e comparada com
%outras em quatro cenários realistas da nossa autoria, simulados no exterior, e os
%resultados são apresentados e discutidos.

\paragraph{}\textbf{Palavras-Chave}: SLAM, Optimização de Hiperparâmetros, faster-lio, fast-livo2, Recuo Simulado 
