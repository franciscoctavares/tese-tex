\paragraph{}A parametrização de soluções de Localização e Mapeamento Simultâneo (SLAM) tem potenciais benefícios para aplicações de robótica móvel em diversos ambientes. O desempenho de sistemas SLAM é altamente dependente da seleção dos parâmetros internos, que influenciam a precisão da trajetória, robustez e eficiência computacional. Consequentemente, Algoritmos de optimização de parâmetros de sistemas SLAM podem ser ferramentas valiosas no estudo da influência dos parâmetros de um sistema SLAM, bem como na optimização da trajetória e redução de erro em aplicações de robótica móvel. 

\paragraph{}Este trabalho consiste na implementação e teste de três algoritmos de optimização de parâmetros para sistemas SLAM: Pesquisa em grelha (Grid Search), Pesquisa aleatória (Random Search) e \textit{Simulated Annealing}. Os algoritmos foram escolhidos tendo em conta a diversidade de algoritmos estudados e o tempo disponível para a sua implementação. Os algoritmos podem ser usados para optimizar qualquer solução de \ac{SLAM} disponível, embora os testes sejam feitos com Faster-LIO e Fast-LIVO2, duas soluções de \ac{SLAM} muito populares na comunidade.

\paragraph{}Neste documento é apresentada uma revisão do estado da arte, identificando as principais lacunas de pesquisa e fornecendo uma contribuição científica com o desenvolvimento deste trabalho. Os algoritmos desenvolvidos foram testados em duas soluções de \ac{SLAM}. Os resultados mostram o potencial do \textit{Simulated Annealing} em relação aos restantes algoritmos implementados, conseguindo em certos casos reduzir o erro da trajetória em mais de metade. Por último, são apresentadas sugestões para melhorias futuras.

\paragraph{}\textbf{Palavras-Chave}: SLAM, Optimização de Hiperparâmetros, Faster-LIO, Fast-LIVO2, \textit{Simulated Annealing}
